\part*{Conclusion}
\addcontentsline{toc}{part}{Conclusion}
\markboth{Conclusion}{Conclusion}

-> Meilleure préparation des données 
-> données plus hétérogènes
-> fomat pivot XML TEI avec un schéma plus complet

- Conditions vers le TAL


- Stage Lectaurep : dialogue entre le chercheur, le métier (archiviste) et les technologies numériques 
- Dans la peau d'un ingénieur en humnités numériques apport des réflexions techniques, des avantages de tel format, modélisation d'environnement et conception 

Partie II : 
- reste intégration dans eScriptorium 
- une fois le serveur IIIF configuré tester l'agrégation des liens IIIF dans le format pivot
- réfléchir à un référentiel des mains

Partie III : 

- mattermost et session de code d'Alix sur Python chaque semaine 2 rdv réguliers

Perspectives professionnelles : 
- Recrutement en Novembre => continuation sur Lectaurep 


Pour l'entrainement d'un segmenteur, il m'a par exemple dit qu'il fallait compter 300 à 400 images segmentées et au moins 100 époques d'entrainement pour avoir quelque chose de vraiment efficace
11:09 AM
donc pour LECTAUREP, ça veut dire qu'on a encore du boulot !