\part*{Conclusion}
\addcontentsline{toc}{part}{Conclusion}
\markboth{Conclusion}{Conclusion}

Arrivé au terme de ce mémoire, quel bilan puis-je tirer, en terme de limites et d'avantages quant aux outils et réflexions que j'ai produits au cours de ce stage ?\\

La présentation de la chaîne de traitement Lectaurep a révélé des structures de données complexes et difficilement interopérables entre-elles, en l'état. De plus nous avons souligner la limite des outils comme \textit{eScriptorium} qui nécessitent encore beaucoup de fonctionnalités et de maintenance.\\

Le modèle commun respectant le standard de la TEI que nous avons mis en place durant le stage est une solution à l'interopérabilité et à la récupération des données, comme les images, au sein d'\textit{eScriptorium}. En posant un cadre de développement à travers une ODD et un script \textit{Generator Lectaurep-TEI}, nous laissons un espace de travail ouvert afin d'affiner les réflexions futures sur l'architecture du canevas XML-TEI. En effet, ce format doit encore accueillir de nouvelles données comme les entités nommées, la structure logique des tableaux des répertoires de notaires et liens IIIF vers les images. Toutes ces informations restent à encoder et feront nécessairement évoluer la structuration XML-TEI proposée dans ce mémoire.\\

Des essais d'intégration dans la plate-forme \textit{eScriptorium} et des projet d'éditorialisation des répertoires à partir du fichier pivot XML-TEI, permettront d'avoir des retours plus précis sur les besoins spécifiques des utilisatrices et des utilisateurs d'\textit{eScriptorium}.\\

L'outil \textit{Kraken-benchmark} est actuellement utilisable pour effectuer les futurs tests des modèles de transcription. Cependant, si l'outil doit encore évoluer et passer à des usages à grande échelle, des optimisations techniques devront être réalisées. En priorité, une revue du code-source, la mise en place de tests unitaires, et l'intégration d'une fonctionnalité permettant d'intégrer des modèles de segmentation de Lectaurep. Ce sont là les conditions pour un éventuel portage de l'application sur un serveur de production ou éventuellement comme une extension (\textit{add-on}) d'\textit{eScriptorium}.\\

Si les tests spécifiques effectués pour les données de Lectaurep ont pu être réalisés avec \textit{Kraken-benchmark}, cependant la durée du stage ne m'a pas permis de réaliser une intégration des résultats dans le fichier pivot XML-TEI. De plus, les résultats mauvais, dû à des modèles défaillants et à la non prise en compte des modèles de segmentation propres à Lectaurep, doivent faire réfléchir aux stratégies actuelles d'entraînement des modèles par Lectaurep.\\ 

Pour la suite du projet, et en s'appuyant sur les principes élémentaires du \textit{Deep learning} que nous nous sommes efforcé de résumer dans ce mémoire, les futurs modèles qui devront disposer de données : plus nombreuses - on estime qu'un modèle de segmentation efficace doit compter pas de moins trois cents à quatre cents pages et plus de cent itérations - et plus hétérogènes afin de travailler sur des jeux de données mixtes. Enfin, une réflexion doit être menée, par le DMC, pour formaliser des règles de transcription pour les annotateurs et les annotatrices d'\textit{eScriptorium} qui préparent les vérités terrains afin que ces données conservent une cohérence lors des entraînements.\\

Mon stage a été une illustration du dialogue actuel des institutions patrimoniales avec les nouvelles technologies, entre les archivistes du DMC et les ingénieur(e)s d'ALMAnaCH. Des réunions mensuelles permettaient d'illustrer les avancés de mes recherches sur les missions que l'on m'avait confiées. De plus le travail à distance à pu être raccourci par les canaux de communication mis en place (\textit{mattermost}, \textit{zoom}). Grâce à ces outils Alix Chagué, notre responsable de stage, Jean-Damien Généro, stagiaire du master TNAH sur le projet \textit{Time us}, et moi-même pouvions collaborer, lors de deux réunions hebdomadaires, sur les avancées et les tâches à réaliser pour la semaine suivante mais également élargir nos horizons en faisant communiquer les approches des deux projets.\\

J'ai beaucoup appris durant ce stage et complété ma formation initiale. J'ai pu appréhender de nouveaux concepts en programmation (la programmation orientée objet, l'exploitation des structures de données plus avancées entre autres) et parfaire ma syntaxe en Python par le biais des conseils dispensés lors des revues de codes et du tutoriel présenté par Alix Chagué. Mais également concernant l'aspect interopérabilité des données et les multiples usages de la TEI, présentés par Laurent Romary.\\  

À compter du 1$^{er}$ novembre 2020, je poursuivrai les missions débutées durant mon stage à ALMAnaCH. Notamment sur les prochaines étapes de Lectaurep qui consisteront à charger des images à partir de IIIF, mettre ces images à disposition des annotateurs et des annotatrices, et récupérer les métadonnées de traitements dans la SIV des Archives nationales. De plus je rejoindrai, le projet \inquote{NER4Archives}, projet en partenariat avec les Archives nationales, visant à améliorer la description des outils de recherche encodés en EAD, grâce à la reconnaissance des différentes entités identifiables dans les champs correspondants. Ainsi que le projet européen \inquote{EHRI III} (\textit{European Holocaust Research Infrastructure}), où l'objectif est d'intégrer des descriptions archivistiques issues du réseau international et d'unifier les représentations et d'en enrichir les contenus (vocabulaire, prosopographie).\\

Durant ce stage j'ai compris que l'ambition d'un projet alliant le patrimoine et le numérique était de rompre avec les pratiques traditionnelles de la recherche d'informations à direction des publics qui pour la plupart maîtrisent les technologies. Mais il ne faut pas oublier de prendre en compte l'écart qui peut encore être présent dans ce public : l'INSEE dans une récente étude, a révélée que l'illectronisme touche encore 17\% de la population française ; la fracture numérique existe toujours\footnote{Rapport, Vie publique/INSEE, 2019, URL : \url{https://www.vie-publique.fr/en-bref/271657-fracture-numerique-lillectronisme-touche-17-de-la-population}}. Ces outils, basés sur l'intelligence artificielle, ne pourraient se suffirent à eux-mêmes et ne seront réellement \inquote{intelligents} que s'ils sont utilisés. Le défi des institutions patrimoniales pour les années à venir sera d'accompagner les utilisateurs vers ces outils et les chercheuses et chercheurs devront faire de l'informatique, non plus une science auxiliaire de l'histoire, mais bien une pratique à part entière, qui s'inscrit dans les méthodes \inquote{éprouvées} de l'histoire\footnote{\cite{heimburger_faire_2011}}. 

Cet accompagnement à la technologie et l'évolution des méthodes en Sciences humaines et sociales, participent en partie, à la condition du succès d'un projet tel que Lectaurep.

\newpage
\thispagestyle{empty}