\part*{Introduction}
\addcontentsline{toc}{part}{Introduction}
\markboth{Introduction}{Introduction}


\bigskip
La question positionner ses compétences informatiques et quels outils crée venir en soutient à des objectifs futures (Intéropérabilité des données et Evaluation de la transcription HTR) des projets patrimoniaux impliquant une dimension numérique  
Mon stage qui s'est déroulé du au explore a porté sur le projet Lectaurep qui explore les possibilités de l'IA. C'est un projet qui est en partenariat avec le laboratoire Almanach d'Inria. Inria ... Almanach ... 
Comment intervenir dans un rpojet en cours de dév qui implique de trouver des solutions pour contourner des problèmes de maintenance logiciel et de de préparer des outils ou du moins dans penser le fonctionnement pour anticiper  ? 

=> possiblités qu'offre le dev infiormatique pour automatiser des tâches et permettre de prendre du recul sur des résultats. 
Plan : L'esprit d'Almananch est de proposer des solutions parmi ces réalisations ... dans Lectaurep j'ai pu intervenir avec cette expertise informatique, que j'ai obtenir durant mes années d'études, notament sur la création d'un format pivot XML TEI ... et la création d'une application Python  Nous aurons l'occassion de revenir en détail sur ce projet, mais il s'agit ... 

% Inria développe des solutions 
% Nouveaux usages des sources + Prise en compte des potentiels numérique dans la culture : besoin d'un dialogue avec ingénieurs formés aux outils informatiques veille technologique pour proposer les inovations

Les nouvelles pratiques informatique  des historiens souligné par l'émmergence des humanités numériques + perspectives de l'IA appliqué au domaine patrimonial => recours des ingénieurs informatiques pour dialoguer. 
Il s’agit autant de réfléchir à la manière pour la machine d’« apprendre » à déchiffrer des manuscrits qu’à la façon de donner accès aux données produites en fonction d’un modèle interprétatif pertinent. La gestion des données massives, les big data, sont ainsi la toile de fond de ces différents projets.

% Appliquer les technologies de l'IA aux archives => enjeu pour Lectaurep pour son corpus de répertoires de notaires allant de 

% rappeler les phases du projet => marie-laurence 

% Le projet en phase 3, moment expérimental => préparer la transcription + plateforme collaborative 

% 

L'omniprésence des outils et de l'implications des techniques informatiques dans les projets patrimoniaux => indispensables retours d'xp sur les choix technologiques réalisés et faciliter leurs réutilisations.

la société est présentée comme en pleine et nécessaire mue numérique => les projets patrimoniaux mettant en jeu des outils numériques se sont multipliés, comme les publications sur le sujet.



% problématique

-> Dans la mesure ou le projet HTR, suit une chaine de traitement définie mais avec des outils en cours de dévellopement, quels outils et des réflexions peuvent s'insérer et venir en appui pour optimiser ? 

% Plan 

Ce mémoire présente le stage effectué parmi l'équipe d'Almanach ... 
