\part*{Introduction}
\addcontentsline{toc}{part}{Introduction}
\markboth{Introduction}{Introduction}

En mars 2018, le mathématicien Cédric Villani rend public le rapport, issu d'une mission parlementaire, intitulé \inquote{Donner un sens à l'intelligence artificielle : pour une stratégie nationale et européenne}\footnote{Rapport Villinai citer} dans lequel il mène une réflexion détaillé sur l'état de l'art et les atouts de l'intelligence artificielle\footnote{on fait généralement remonter ce terme à 1956 avec l'article...} en France. Dans ce rapport il apparaît que la France compte parmi les quatre premiers pays au monde avec la Chine, les États-Unis et le Royaume-Uni pour la production mondiale d'articles sur l'intelligence artificielle et rend compte d'une définition de l'IA non comme :
\begin{quote}
    un champ de recherches bien défini qu'un programme, fondé autour d'un objectif ambitieux : comprendre comment fonctionne la cognition humaine et la reproduire; créer des processus cognitifs comparables à ceux de l'être humain.\footnote{citer le rapport Villani pp.9}
\end{quote}
Cette définition n'est peut-être pas la plus complète et la seule qui existe\footnote{ pour d'autres citer déf. on les classent générallement suivants les catégories Intelligence Artificielle de Stuart Russel et Norvig, Chap1, pp.4} mais elle possède l'avantage de définir l'IA comme l'imitation parfaite des performances humaines. En 1950, Alan Turing proposait un test afin de savoir si un ordinateur avait acquis l'intelligence opérationnelle d'un humain : après une série de questions posé à l'ordinateur par un humain, le test était réussi si l'humain en question était dans l'incapacité de dire si les réponses provenaient d'un humain ou d'un système informatisé. Dès lors, l'ordinateur pour passer ce test, et se confondre à l'humain, devrai posséder les fonctionnalités suivantes : \textbf{le traitement du langage naturel}, pour communiquer, la \textbf{représentation des connaissances}, sous la forme d'une mémoire, un \textbf{raisonnement automatisé}, tirer des conclusions logiques de l'expérience mémorisé, et l'\textbf{apprentissage}, pour s'adapter aux circonstances et s'adapter aux hasard. Enfin pour simuler entièrement l'humain et passer le test de Turing dit \inquote{complet}, la perception du système pourrai être vérifié à l'aune de la \textbf{vision articielle}, pour percevoir les objets, et la \textbf{robotique} pour les manipuler.\footnote{ces exemples sont tirés de IA pp. 3}. Parmi les domaines et les applications de l'IA les plus citées, les véhicules autonomes avec la voiture robotisé de l'université de Stanford en 2005, la reconnaissance de la parole, avec les assistants personnels comme Alexa (2014), les jeux avec \textit{Deep Blue} d'IBM, le super ordinateur qui a battu le champion mondial Garry Kasparov aux échecs en 1997 puis \textit{Alpha Go} en 2015 de Google Deepmind qui bat le champion du monde du jeu de go, la traduction automatique, avec \textit{Google Translate} en 2006, plus récémment la propagation du virus Covid-19, a accentué les usages de l'IA dans les domaines de la santé pour identifié certains clusters sanitaires localisés avec plus ou moins de succès en se basant sur les données médicales. D'après ces exemples et le rapport Villani évoqué plus haut les secteurs prioritaires de l'IA concerne avant tout : la santé, les transports et l'environnement et la défense. Dès lors, comment identifiée l'actuelle plus-value de l'IA appliquée à la culture aux travers des projets patrimoniaux ?  Comment le projet Lectaurep illustre t-il ce mouvement de la culture vers l'IA ? 
\bigskip

% Nouveaux usages des sources + Prise en compte des potentiels numérique dans la culture : besoin d'un dialogue avec ingénieurs formés aux outils informatiques veille technologique pour proposer les inovations

Les nouvelles pratiques informatique  des historiens souligné par l'émmergence des humanités numériques + perspectives de l'IA appliqué au domaine patrimonial => recours des ingénieurs informatiques pour dialoguer. 
Il s’agit autant de réfléchir à la manière pour la machine d’« apprendre » à déchiffrer des manuscrits qu’à la façon de donner accès aux données produites en fonction d’un modèle interprétatif pertinent. La gestion des données massives, les big data, sont ainsi la toile de fond de ces différents projets.

% Appliquer les technologies de l'IA aux archives => enjeu pour Lectaurep pour son corpus de répertoires de notaires allant de 

% rappeler les phases du projet => marie-laurence 

% Le projet en phase 3, moment expérimental => préparer la transcription + plateforme collaborative 

% 

L'omniprésence des outils et de l'implications des techniques informatiques dans les projets patrimoniaux => indispensables retours d'xp sur les choix technologiques réalisés et faciliter leurs réutilisations.

la société est présentée comme en pleine et nécessaire mue numérique => les projets patrimoniaux mettant en jeu des outils numériques se sont multipliés, comme les publications sur le sujet.



% problématique

-> Dans la mesure ou le projet HTR, suit une chaine de traitement définie mais avec des outils en cours de dévellopement, quels outils et des réflexions peuvent s'insérer et venir en appui pour optimiser ? 

% Plan 

Ce mémoire présente le stage effectué parmi l'équipe d'Almanach ... 
