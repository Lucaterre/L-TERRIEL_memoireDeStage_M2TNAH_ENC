\chapter*{Liste des sigles et abréviations}
\addcontentsline{toc}{chapter}{Liste des sigles et abréviations}
\markboth{Liste des sigles abréviations}{} 

\begin{center}
$\star$ \textbf{\textsc{ORGANISMES}} $\star$
\end{center} 
\begin{itemize}
    \item AN : \emph{Archives Nationales}
    \item ALMANACH : \emph{Automatic Language Modelling and Analysis \& Computational Humanities}
    \item DMC : \emph{Département du Minutier central des notaires de Paris}
    \item DMOASI : \emph{Département de la maîtrise d’ouvrage du système d'information (direction de l'appui scientifique)} 
    \item INRIA : \emph{Institut Nationale de Recherche en Informatique et Automatique}
    \item W3C : \emph{\textit{World Wide Web Consortium}}
\end{itemize}

\begin{center}
$\star$ \textbf{\textsc{DOMAINES ET DISCIPLINES}} $\star$
\end{center} 
\begin{itemize}
    \item IA : \emph{Intelligence Artificielle}
    \item DL : \emph{\textit{Deep Learning}}
    \item ML : \emph{\textit{Machine Learning}}
    \item POO : \emph{\textit{Programmation Orientée Objet}}
    \item REN : \emph{Reconnaissance d'Entités Nommées} (en anglais, NER ou \textit{Name Entity Recognition})
    \item RNN : \emph{Recurrent neural network} (en français, \emph{Réseau de neurones récurrents})
    \item TAL : \emph{Traitement Automatique des Langues} (en anglais, NLP ou \textit{Natural Language Processing})
\end{itemize}

\begin{center}
$\star$ \textbf{\textsc{TECHNOLOGIES}} $\star$
\end{center} 
\begin{itemize}
    \item BASH : \emph{Bourne-Again shell}
    \item CMS : \emph{Content Management System} (en français, \emph{système de gestion de contenu})
    \item CSS : \emph{Cascading Style Sheets}
    \item CSV : \emph{Comma-separated values}
    \item DTD : \emph{Document Type Definition}
    \item HTML : \emph{HyperText Markup Language}
    \item HTTP : \emph{Hypertext Transfer Protocol}
    \item HTR : \emph{Handwritten Text Recognition}
    \item IIIF : \emph{International Image Interoperability Framework}
    \item JSON : \emph{JavaScript Object Notation}
    \item JSON-LD : \emph{JavaScript Object Notation for Linked Data}
    \item OCR : \emph{Optical Character Recognition}
    \item ODD : \emph{One Document Does it all}
    \item OS : \emph{Operating System} (en français, \emph{Système d'exploitation})
    \item PDF : \emph{Portable Document Format}
    \item RDF : \emph{\textit{Ressource Description Framework}}
    \item RELAXNG : \emph{Regular Language for XML Next Generation}
    \item SGML : \emph{Standard Generalized Markup Language}
    \item TEI : \emph{Text Encoding Initiative}
    \item WSGI : \emph{Web Server Gateway Interface}
    \item XML : \emph{eXtensible Markup Language}
    \item XSLT : \emph{eXtensible Stylesheet Language Transformations}
\end{itemize}

\begin{center}
$\star$ \textbf{\textsc{STANDARDS, NORMES, RÉFÉRENTIELS ET ONTOLOGIES}} $\star$
\end{center} 
\begin{itemize}
    \item ALTO : \emph{Analysed Layout and Text Object}
    \item EAC-CPF : \emph{Encoded Archival Context - Corporate Bodies, Persons and Families} 
    \item EAD : \emph{Encoded Archival Description} 
    \item EXIF : \emph{EXchangeable Image File Format}
    \item ISAD(G) : \emph{International Standard Archival Description-General} (en français, \emph{Norme générale et internationale de description archivistique})
    \item ISAAR(CPF) : \emph{International Standard Archival Authority Record for Corporate Bodies, Persons and Families} (en français, \emph{Norme internationale sur les notices d'autorité archivistiques relatives aux collectivités, aux personnes et aux familles})
    \item RIC-O : \emph{Records in Contexts-Ontology}
    \item RIC-CM : \emph{Records in Contexts-Conceptual Model}
    \item EXIF : \emph{EXchangeable Image File Format}
    \item TEI : \emph{Text Encoding Initiative} 
\end{itemize}

\begin{center}
$\star$ \textbf{\textsc{MÉTRIQUES}} $\star$
\end{center} 
\begin{itemize}
    \item CER : \emph{Character Error Rate} (en français, \emph{Taux d'erreur de caractères})
    \item WACC : \emph{Word Accuracy} (en français, \emph{Taux de reconnaissance de mots})
    \item WER : \emph{Word Error Rate} (en français, \emph{Taux d'erreur de mots})
\end{itemize}

\begin{center}
$\star$ \textbf{\textsc{AUTRES}} $\star$
\end{center} 
\begin{itemize}
    \item EN : \emph{Entités Nommées}
    \item PEP : \emph{Python Enhancement Proposals}
    \item SIV : \emph{Salle des inventaires virtuels} 
\end{itemize}

\newpage
\thispagestyle{empty}