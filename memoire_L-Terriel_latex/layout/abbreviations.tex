\chapter*{Liste des sigles et abréviations}
\addcontentsline{toc}{chapter}{Liste des sigles et abréviations}
\markboth{Liste des sigles abréviations}{} 

\begin{itemize}
    \item A.N. : Archives Nationales
    \item DMC : Département du Minutier central des notaires de Paris  
    \item DMOASI : département de la maîtrise d’ouvrage du système d’information (direction de l’appui scientifique) 
    \item SIV : Salle des inventaires virtuels 
\end{itemize}

\begin{center}
$\star$
\end{center} 

\begin{itemize}
    \item ALMAnaCH : \emph{Automatic Language Modelling and Analysis \& Computational Humanities}
    \item ANR : Agence Nationale de la Recherche
    \item EPI : Équipe-Projet Inria
    \item INRIA : Institut Nationale de Recherche en Informatique et Automatique
    \item LECTAUREP : Lecture automatique des répertoires
    \item W3C : World Wide Web Consortium
\end{itemize}

\begin{center}
$\star$
\end{center} 

\begin{itemize}
    \item IA : Intelligence Artificielle 
    \item EN : Entités Nommées
    \item ML : \emph{Machine Learning}
    \item REN : Reconnaissance d'Entités Nommées
    \item RNN : \emph{Recurrent neural network} - Réseau de neurones récurrents
    \item TAL : Traitement Automatique des Langues
\end{itemize}


\begin{center}
$\star$
\end{center} 

\begin{itemize}
    
    \item CSS : \emph{Cascading Style Sheets}
    \item CSV : \emph{Comma-separated values}
    \item DTD : \emph{Document Type Definition}
    \item HTML : \emph{HyperText Markup Language}
    \item HTR : \emph{Handwritten Text Recognition}
    \item OCR : \emph{Optical Character Recognition}
    \item ODD : \emph{One Document Does it all}
    \item PDF : \emph{Portable Document Format}
    \item RELAXNG : \emph{Regular Language for XML Next Generation}
    \item TEI : \emph{Text Encoding Initiative}
    \item XML : \emph{eXtensible Markup Language}
    \item XSLT : \emph{eXtensible Stylesheet Language Transformations}
\end{itemize}