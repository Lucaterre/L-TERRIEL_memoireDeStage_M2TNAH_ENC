\chapter*{Remerciements}
\addcontentsline{toc}{chapter}{Remerciements}
\markboth{Remerciements}{} 

Pour ce stage, qui s'est déroulé durant la période de confinement, je tiens à remercier l'ensemble des personnes qui m'ont aidé et soutenu dans cette situation particulière et qui m'ont encouragé dans mon travail.

En premier lieu, je remercie mon tuteur pédagogique, M. Thibault Clérice, et ma tutrice professionnel, Mme Alix Chagué, ingénieure recherche et développement, pour leurs appuis et leurs conseils avisés. 

Je remercie l'ensemble de l'équipe du projet Lectaurep, et le personnel du Minutier central des notaires de Paris aux Archives nationales, Mme Marie-Françoise Limon-Bonnet, responsable du département, Mme Aurélia Rostaing, responsable du pôle instruments de recherche, M. Gaetano Piraino, responsable à la DMOASI, M. Danis Habib, chargé d'études documentaires, Mme Virginie Grégoire, secrétaire de documentation, M. Benjamin Davy, agent technique d'accueil, surveillance et magasinage, et Mme Anna Chéru, stagiaire phase 3. 

Je remercie l'équipe ALMAnaCH d'Inria, qui ont su créer les conditions favorables d'accueil de mon stage et permis un environnement de travail stimulant; je remercie en particulier M. Laurent Romary, directeur de recherche, pour nos échanges et ses conseils pertinents et Mme Florianne Chiffoleau, ingénieure recherche et développement, pour son aide.

Comme le stage est un travail d'équipe avant tout, je remercie Jean-Damien Généro, collègue de master TNAH à l'École nationale des chartes, qui était stagiaire durant la même période à ALMAnaCH sur le projet Time us, dont la mise en commun de nos efforts et nos échanges sur les scripts ont permis le bon déroulement technique et pratique de nos stages respectifs.

Enfin je remercie ma famille et mes amis, pour leurs indéfectible soutient tout au long de cette période. 