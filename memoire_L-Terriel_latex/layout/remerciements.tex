\chapter*{Remerciements}
\addcontentsline{toc}{chapter}{Remerciements}
\markboth{Remerciements}{} 

Pour ce stage, qui s'est déroulé durant la période de confinement, je tiens à remercier l'ensemble des personnes qui m'ont aidé et soutenu dans cette situation particulière et qui m'ont encouragé dans mon travail.

Je remercie ma tutrice professionnelle, Mme Alix Chagué, et mon tuteur pédagogique, M. Thibault Clérice, pour leurs appuis et leurs conseils avisés (Merci Alix pour tes conseils \inquote{pythoniques}...).

Je remercie l'équipe ALMAnaCH d'Inria, qui a su créer des conditions favorables d'accueil pour mon stage et permis un environnement de travail stimulant; je remercie tout particulièrement M. Laurent Romary, directeur de recherche, pour nos échanges et conseils pertinents sur la TEI et Mme Florianne Chiffoleau, ingénieure recherche et développement, pour son aide.

Comme le stage est un travail d'équipe, je remercie Jean-Damien Généro, collègue du master TNAH à l'École nationale des chartes, qui était également stagiaire sur la même période à ALMAnaCH sur le projet \inquote{Time us}. Nos échanges et la mise en commun de nos travaux ont contribué à rendre le stage encore plus enrichissant.

Je remercie l'ensemble de l'équipe du projet Lectaurep, et le personnel du Minutier central des notaires de Paris aux Archives nationales pour leur disponibilité : Mme Marie-Françoise Limon-Bonnet, responsable du département, Mme Aurélia Rostaing, responsable du pôle instruments de recherche, M. Gaetano Piraino, responsable à la DMOASI, M. Danis Habib, chargé d'études documentaires, Mme Virginie Grégoire, secrétaire de documentation, M. Benjamin Davy, agent technique d'accueil, surveillance et magasinage, et Mme Anna Chéru, stagiaire en phase 3. 

Enfin je remercie, ma famille, mes camarades de promotions : Duchesse Anne Brunet, Mathilde Daugas, Edward Gray PhD, Chloë Fize (11101010 10011110 10101001), Morgane Rousselot ; mes amis de longue date : Léa Mièle et Benjamin Luzu, pour leur indéfectible soutien tout au long de cette période. 
\newpage
\thispagestyle{empty}