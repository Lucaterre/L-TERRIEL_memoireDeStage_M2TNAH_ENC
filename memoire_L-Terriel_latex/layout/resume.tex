\chapter*{Résumé}
\addcontentsline{toc}{chapter}{Résumé}
\markboth{Résumé}{} 
Ce mémoire a été réalisé dans le cadre de l'obtention du diplôme de Master 2\up{ème} année \inquote{Technologies numériques appliquées à l'histoire} de l'École nationale des chartes. 
Il est rédigé dans le contexte d'un stage de quatre mois au sein de l'équipe projet ALMAnaCH de INRIA, et dont l'action s'est inscrite dans le cadre de la phase 3 du projet Lectaurep, porté par le Ministère de la Culture, les Archives Nationales, et Scripta portant sur la reconnaissance des écritures manuscrites et l'extraction d'informations sur les répertoires de notaires (1803-1944). Le projet met en \oe{}uvre les méthodes récentes d'apprentissage machine qui utilisent la plate-forme de transcription \textit{eScriptorium}, interface graphique du système HTR à base de réseaux de neurones artificielles \textit{Kraken}. Il parcourt les tentatives de constitution d'un fichier pivot XML-TEI afin de structurer les données entrantes et sortantes de la plate-forme \textit{eScriptorium}, notamment lors de la récupération d'images. Il examine ensuite les étapes de recherche de métriques pour comparer des chaînes de texte et du développement d'une application Python dédiée à l'évaluation des modèles de transcription \textit{Kraken-Benchmark}, testée sur les données de Lectaurep. Ces métriques pouvant être intégrées dans le \textit{workflow} général du projet et dans le fichier pivot XML-TEI. Ce mémoire s'attache à montrer comment ces axes pourraient être améliorés par la suite. Il s'agit d'une présentation des stratégies, des choix critiques et des enjeux envisagés durant le stage, dont l'objectif est de rendre compte des réalisations techniques et des choix scientifiques opérés dans un contexte mêlant le patrimoine et les enjeux numériques. 

\bigskip
\textbf{Mots-clefs:} Lectaurep; Archives nationales; Minutier central; INRIA; Répertoires de notaires; XIX$^{e}$ siècle; XX$^{e}$ siècle; HTR; Format pivot; XML; TEI; ALTO; EAD; EAC; EXIF; ODD; XSLT; Développement applicatif; Python; métriques; Intelligence artificielle; Réseaux de neurones; \textit{data science}; Apprentissage machine; Similarité syntaxique; Similarité sémantique; TAL; Humanités numériques

\bigskip
\textbf{Informations bibliographiques:} Lucas Terriel, \textit{Représenter et évaluer les données issues du traitement automatique d'un corpus de documents historiques. L'exemple de la reconnaissance des écritures manuscrites dans les répertoires de notaires du projet LectAuRep.}, mémoire de master \og Technologies numériques appliquées à l'histoire \fg{}, dir. Alix Chagué et Thibault Clérice, École nationale des chartes, 2020.

\clearpage
\thispagestyle{empty}
\cleardoublepage