\chapter*{Résumé}
\addcontentsline{toc}{chapter}{Résumé}
\markboth{Résumé}{} 
Ce mémoire a été réalisé dans le cadre de l'obtention du diplôme du Master de 2\up{ème} année \inquote{Technologies numériques appliquées à l'histoire} de l'École nationale des chartes. 
Il a été rédigé dans le contexte d'un stage de quatre mois à INRIA au sein de l'équipe projet ALMAnaCH en soutien au projet Lectaurep.
Ce projet, débuté en 2018, est porté par diverses institutions : Ministère de la Culture (DIN), les Archives Nationales (DMC/DMOASI), Scripta (EPHE) et INRIA (ALMAnaCH) qui jouent chacune un rôle clé. 
Bénéficiant du double mouvement de numérisation massive des documents de la part des institutions patrimoniales, permettant une plus grande accessibilité aux documents et d'une refonte épistémologique des sciences humaines par le paradigme de l'informatique que sont les \inquote{humanités numériques}, Lectaurep ambitionne de repenser l'usage des quelques deux milles répertoires de notaires (1803-1944) par le biais des techniques d'\inquote{intelligence artificielle} appliqués aux textes (TAL) comme la reconnaissance automatique de la structure et d'écritures manuscrites (HTR) et les techniques de détection d'informations (NER). 
La plateforme eScriptorium (Scripta/EPHE), doit permettre à terme la mise en place d'un éco-système complet pour envisager la correction collaborative des transcriptions (\textit{crowdsourcing}), l'annotation, l'indexation, la publication et la récupération de métadonnées détaillées.
L'auteur, qui a participé à la phase 3 du projet, présente dans ce mémoire la mise en place d'un format pivot XML TEI pour représenter et homogénéiser les données et métadonnées dans des formats hétérogènes présentent dans la chaîne de traitement du projet ainsi que l'évaluation des premiers modèles de transcriptions, entraînés avec Kraken, par le biai de la conception \textit{ex-nihilo} d'une application web en Python générique qui présente en sortie un tableau de métriques (\textit{dashboard}) de types \textit{text-to-text} pour permettre la confrontation d'un document de vérité terrain et de sa prédiction par le système OCR.
L'auteur s'attache également a évoquer les liens entre ces deux missions, en apparence éloignées, et à envisager des perspectives d'optimisation pour le développement des outils présentés.\\
\bigskip
%informations à compléter
\textbf{Mots-clefs:} Lectaurep; Archives nationales; Minutier central; INRIA; Répertoires de notaires; XIXe siècle; XXe siècle; Format pivot; XML; TEI; ALTO; EAD; EAC; EXIF; ODD; XSLT; Développement applicatif; Python; Flask; CLI; HTR; Reconnaissance automatique; Kraken; Intelligence artificielle; Réseaux de neurones; Écosystème data science; Apprentissage machine; Similarité syntaxique et sémantique; Évaluation \textit{text-to-text}; Transcription; Traitement automatique du langage; Humanités numériques

\bigskip
\bigskip
\bigskip

% informations à compléter
\textbf{Informations bibliographiques:} Lucas Terriel, \textit{Représenter et évaluer les données issues du traitement automatique d'un corpus de documents historiques. L'exemple de la reconnaissance des écritures manuscrites dans les répertoires de notaires du projet LectAuRep.}, mémoire de master \og Technologies numériques appliquées à l'histoire \fg{}, dir. Alix Chagué et Thibault Clérice, École nationale des chartes, 2020.

\clearpage
\thispagestyle{empty}
\cleardoublepage