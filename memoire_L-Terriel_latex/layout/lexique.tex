\chapter*{Lexique informatique}
\addcontentsline{toc}{chapter}{Lexique informatique}
\markboth{Lexique informatique}{} 

\textit{\small{liste non exhaustive}}

\begin{itemize}
    \item \textbf{Algorithme} : Suite d'étapes réalisées par un programme informatique pour effectuer une tâche.
    \item \textbf{\textit{Back-office/Front-office}} : Dans une application désigne la partie visible par le client (\textit{front-office}) et la partie qui concerne les systèmes d'informations (bases de données) et leurs gestion invisible de l'utilisateur final (\textit{back-office}).
    \item \textbf{\textit{Blog}} : Type de site \textit{web} qui permet la publication périodique d'articles scientifiques rendant compte de l'actualité d'un projet ou d'une thématique.
    \item \textbf{Interface en ligne de commande} : CLI ou \textit{Command Line Interface} en anglais - c'est une interface homme machine dépourvue d'aspect graphique et où la communication s'effectue en mode texte grâce à des lignes de commande (texte sur le clavier) pour demander à l'ordinateur d'effectuer une opération.
    \item \textbf{\textit{Commit}} : Dans un système de versionnage, commande qui permet de valider des modifications locales vers un référentiel central afin de les mettre à disposition.
    \item \textbf{Depôt (informatique)} : \textit{repository} en anglais -  S'applique aux logiciels de gestion de versions, stockage organisé de données; endroit où l'on dépose le code-source.
    \item \textbf{\textit{Docstring}} : Chaîne de caractères pour documenter un segment spécifique du code informatique.
    \item \textbf{Fonction} : En programmation, \inquote{sous-programme} qui permet de réaliser des tâches répétitives pour alléger du code.
    \item \textbf{Interface graphique} : GUI ou \textit{Graphical User Interface} en anglais - est un environnement qui permet l'interaction de l'homme et la machine comme la métaphore du bureau dans la plupart des systèmes d'exploitation.
    \item \textbf{Interpréteur de commande} : Terminal informatique, c'est un logiciel étant compris de base dans le système d'exploitation, il permet d'interpréter des commandes d'un utilisateur ou d'une utilisatrice dans un environnement dépourvu d'interface graphique.
    \item \textbf{\textit{Issue}} : Dans une plateforme de versionnage, sorte de billet qui permet d'émettre des suggestions ou qui fait état des bugs dans un contexte de développement. 
    \item \textbf{\textit{Merge}} : Action de fusionner des branches (versions) différentes d'un dépôt informatique. (Cf. \textit{pull request} (Github) ou \textit{merge request} (Gitlab))
    \item \textbf{module} : En développement, fichier contenant des fonctions, des classes ou des variables pouvant être importé dans un \textit{script} pour en utiliser le contenu.
    \item \textbf{\textit{Package}} - sym. \textit{library} : Ensemble de modules de traitements spécifiques pouvant être importé.
    \item \textbf{\textit{Parser}} :  Programme informatique qui permet l'analyse syntaxique des éléments pour leurs donner une signification. 
    \item \textbf{\textit{Push}} : Dans un système de versionnage, commande qui après un \textit{commit} permet d'envoyer les modifications d'un système local vers un référentiel central pour partager ces dernières.
    \item \textbf{\textit{Open-source}} : Communauté et types de licences qui s'appliquent à des programmes ou à des logiciels qui permettent la redistribution et la réutilisation de leurs code-sources.  
    \item \textbf{\textit{Script}} : Programme ou extrait de programme qui permet de réaliser une tâche prédéfinie (Cf. Algorithme). 
    \item \textbf{Système d'exploitation} : \textit{Operating System} (OS) en anglais - Ensemble de programmes qui permet d'utiliser les ressources d'un ordinateur. Le logiciel système pilote les ressources matérielles de l'ordinateur et reçoit les instructions des usagers ou d'autres logiciels.
    \item \textbf{Logiciel de gestion de versions} : Logiciel qui permet de stocker des fichiers en conservant la chronologie de l'ensemble des modifications (versions) qui y ont été effectuées.
    \item \textbf{Tests unitaires} : En programation informatique, procédure qui permet de vérifier une partie précise d'un logiciel ou d'un programme pour s'assurer de son fonctionnement.
\end{itemize}