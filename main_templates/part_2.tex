\part{Représenter, enrichir et homogénéiser dans un format pivot les données et métadonnées au sein de la chaîne de traitement Lectaurep}

Objectifs de la mission ? 

\chapter{Enjeux et problématiques liés aux données et aux formats dans le projet}
\section{De l'importance des données et métadonnées}
\section{Les répertoires de notaires ne sont pas que des images numérisées !}
- permettre IIIF

\chapter{Le choix du XML TEI (\textit{Text Encoding Initiative}) comme format pivot}
\section{Un choix réaliste ? le format XML TEI dans d'autres projets et apports pour Lectaurep}
\section{Esquisse d'un format pivot et premières spécifications TEI grâce l'ODD}

\chapter{Simuler la récupération, l'export et la validation d'un format pivot XML TEI}
\section{Objectifs et buts de la simulation}
- ne parle forcément aux archivistes;
- besoin de visualiser les données dans un canevas TEI et d'envisager;
\section{Un CLI en Python pour générer un format pivot XML TEI et valider grâce à un schéma RELAX NG}