% --------------------------------------------------------------------------------------
%% Canevas pour le mémoire de Master 2 TNAH - ENC (10/08/2020)
%% main.tex
%% Copyright 2020 L. Terriel (lucas.terriel@chartes.psl.eu / ls.terriel@gmail.com)
% --------------------------------------------------------------------------------------
% Ce(tte) œuvre est mise à disposition selon les termes de la Licence Creative Commons :
%% - Attribution 
%% - Pas d’Utilisation Commerciale 
%% - Partage dans les Mêmes Conditions 4.0 International.
% --------------------------------------------------------------------------------------
% Ce fichier fait partie d'un ensemble de fichiers .tex 
% appartenant au projet LaTeX "L-TERRIEL_memoireDeStage_M2TNAH_ENC"
% --------------------------------------------------------------------------------------

\documentclass[a4paper, twoside, 12pt]{book}

\usepackage[french]{babel}

%--------------------------
% INPUTENC (encodage du texte)
% FONTENC (positionnement des accents)
%--------------------------
\usepackage[utf8]{inputenc}
\usepackage[T1]{fontenc}
%--------------------------
% HYPERREF (liens hypertextes et métadonnées)
%--------------------------
\usepackage{hyperref}
\hypersetup{
colorlinks=true,
linkcolor=black,
urlcolor=blue,
citecolor=black
}

\title{Mémoire de Master 2 Technologies numériques appliqués à l'histoire - Ecole nationale des chartes - 2020}
\author{Lucas Terriel}

%--------------------------
% TOCBIBIND (ajouter la bibliographie dans la Table des matières)
%--------------------------
\usepackage{tocbibind}

%--------------------------
% Éléments de mise en page (marge de 2,5 cm, alinéa en début de paragraphe 1cm, interligne 1,5)
%--------------------------
\usepackage[a4paper, margin=2.5cm]{geometry}
\usepackage{setspace}
\onehalfspacing
\setlength{\parindent}{1cm}

\usepackage{fancyhdr}
\renewcommand{\chaptermark}[1]{\markboth{#1}{}}
\renewcommand{\sectionmark}[1]{\markright{#1}}
\pagestyle{fancy}
\fancyhf{}
\fancyhead[LE,RO]{\thepage}
\fancyhead[LO]{\nouppercase{\rightmark}}
\fancyhead[RE]{\nouppercase{\leftmark}}
\renewcommand{\headrulewidth}{0pt}

%--------------------------
% Mise en page des tableaux
%--------------------------

% pour les longs tableaux
\usepackage{longtable}
% longueur / largeur :
\usepackage{chngpage}
% paysage :
\usepackage{lscape} 
% couleurs
\usepackage{colortbl}

%--------------------------
% Modules pour gérer les listes
%--------------------------
\usepackage{enumerate}
\usepackage{enumitem}

%--------------------------
% Modules pour créer des arborescences 
%--------------------------
\usepackage{dirtree}

%-------------------------
% Modules pour citer du code (LISTING)
%-------------------------
\usepackage{listings}
\usepackage{xcolor}

\definecolor{dkgreen}{RGB}{112,211,98}
\definecolor{gray}{rgb}{0.5,0.5,0.5}
%\definecolor{mauve}{rgb}{0.58,0,0.82}
\definecolor{orange}{RGB}{229,148,0}
%\definecolor{lightblue}{RGB}{50,162,223}
%\definecolor{brickred}{RGB}{206,28,9}
\definecolor{dkblue}{RGB}{71,118,105}
\definecolor{lightgray}{RGB}{251,252,251}

\lstset{%
backgroundcolor=\color{lightgray},
frameround=fttt,
aboveskip=3mm,
belowskip=3mm,
showstringspaces=false,
columns=flexible,
basicstyle={\small\ttfamily},
numbers=left,
numberstyle=\tt\tiny\color{gray},
keywordstyle=\color{dkblue},
commentstyle=\color{dkgreen},
stringstyle=\color{orange},
breaklines=true,
breakatwhitespace=true,
tabsize=3
}

%--------------------------
% GESTION DE LA BIBLIOGRAPHIE
%--------------------------
\usepackage[backend=biber, sorting=nyt, style=enc]{biblatex}
\usepackage[autostyle]{csquotes}

%\bibliography{bibliographie.bib}
\addbibresource{bibliographie.bib}

%--------------------------
% COMMANDES ET ENVIRONNEMENTS PERSONNALISÉS
%--------------------------
\newcommand{\hugeskip}{\bigskip \bigskip \bigskip}
\newcommand{\citecode}[1]{\texttt{\textmd{#1}}}
\newcommand{\inquote}[1]{\og{}#1\fg{}}

\setcounter{secnumdepth}{5}
\setcounter{tocdepth}{2}

\usepackage[official]{eurosym}
\usepackage{afterpage}

%--------------------------
% FIGURES
%--------------------------
\usepackage{graphicx}
\graphicspath{ {./images/} }

\usepackage{float}

% ============================================

\begin{document}

% Pour les parties réglementaires : 
\frontmatter
\begin{titlepage}
    \begin{center}

        \bigskip
    
        \begin{large}
            \'ECOLE NATIONALE DES CHARTES 
        \end{large}
    
        \begin{center}
            \rule{4cm}{0.02cm}
        \end{center}
    
        \hugeskip
        
        \begin{Large}
             \textbf{Lucas Terriel}\\
        \end{Large}
        \begin{normalsize}
            \textit{Licencié ès histoire}\\
            \textit{Diplômé de master 
            Histoire, Civilisations, Patrimoine contemporains}\\
        \end{normalsize}
        
        \hugeskip
        \bigskip
        
        % informations à compléter
        \begin{LARGE}
            \textbf{Représenter et évaluer les données issues de la structuration et de la transcription automatique d'un corpus.}
            
        \end{LARGE}
        \bigskip
        \bigskip
        \bigskip
        \bigskip
        \begin{large}
            \textbf{L'exemple de la reconnaissance automatique des écritures manuscrites sur les répertoires de notaires du projet Lectaurep.}\\
        \end{large}
        
        \hugeskip
        \vfill
        
        
        
        \begin{large}
            Mémoire pour le diplôme\\
            \og Technologies numériques appliquées à l'histoire \fg\\
            \bigskip
            2020
        \end{large}
        
        \begin{figure}[H]
            \centering
            \includegraphics[width=2.5cm]{cc-icon.png}
            \label{traitement}
        \end{figure}
        
    \end{center}
\end{titlepage}

\thispagestyle{empty}
\cleardoublepage
        \begin{figure}[H]
            \centering
            \includegraphics[width=3cm]{CC-BY-NC-logo.png}
            \label{traitement}
        \end{figure}

    Ce mémoire professionnel/recherche est placé sous les termes de la licence Creative Commons en ces termes : \textbf{Licence Creative Commons Attribution - Pas d'Utilisation Commerciale - Partage dans les Mêmes Conditions 4.0 International (CC BY-NC 4.0).)}.
\bigskip

    Consulter la \href{https://creativecommons.org/licenses/by-nc-sa/4.0/legalcode.fr}{licence\footnote{
\textit{Licence CC BY-NC 4.0 International}, en ligne: \url{https://creativecommons.org/licenses/by-nc/4.0/legalcode.fr}.}} en entier pour plus de détails. 
\bigskip

    Vous êtes autorisés à :
    \begin{itemize}
        \item \textbf{Attribution} : Vous devez créditer l'\oe uvre, intégrer un lien vers la licence et indiquer si des modifications ont été effectuées à l'\oe uvre. Vous devez indiquer ces informations par tous les moyens raisonnables, sans toutefois suggérer que l'Offrant vous soutient ou soutient la façon dont vous avez utilisé son \oe uvre;
        \item \textbf{Pas d'utilisation commerciale} : Vous n'êtes pas autorisé à faire un usage commercial de cette \oe uvre, tout ou partie du matériel la composant. 
        \item \textbf{Partager - copier, distribuer et communiquer} le matériel par tous moyens et sous tous formats;
        \item \textbf{Adapter - remixer, transformer et créer} à partir du matériel.
    \end{itemize}
\bigskip

\bigskip
\bigskip
\bigskip
\bigskip
\bigskip
\bigskip
\bigskip
\bigskip
\bigskip
\bigskip
\bigskip
\bigskip
\bigskip
\bigskip
\bigskip
\bigskip
\bigskip
\bigskip
\bigskip

\newpage
\thispagestyle{empty}
\chapter*{Résumé}
\addcontentsline{toc}{chapter}{Résumé}
\markboth{Résumé}{} 
Ce mémoire a été réalisé en vue de l'obtention du diplôme de Master 2 \inquote{Technologies numériques appliquées à l'histoire} de l'École nationale des chartes. Il a été rédigé à la suite d'un stage de quatre mois au sein de l'équipe ALMAnaCH d'Inria, et dont le déroulé s'est inscrit dans le cadre de Time Us. Ce projet de recherche pluri-institutionnel porte sur l'histoire de l'industrie du textile en France (fin XVII\up{e}-début XX\up{e} siècles) et sur la reconstitution des budget temps des ouvriers et ouvrières du textile. Time Us explore également l'utilisation d'outils informatique pour réaliser cette recherche. 
Ce mémoire étudie le traitement informatique appliqué à une partie du corpus de Time Us, de la structuration à l'éditorialisation. 
Il s'agit d'une analyse critique des enjeux, stratégies et résultats envisagés dans le cadre du projet Time Us autant que du stage, dont le but est de rendre compte d'un exemple de projet et de développement s'inscrivant dans le cadre des humanités numériques.


\bigskip

%informations à compléter
\textbf{Mots-clefs:} TAL; métadonnées; données; format pivot; XML-TEI; développement applicatif; similarité syntaxique; similarité sémantique; métriques \textit{text-to-text}; OCR; HTR; Machine learning; Intelligence artificielle; répertoires de notaires; valorisation patrimoniale; Humanités numériques.

\bigskip
\bigskip
\bigskip

% informations à compléter
\textbf{Informations bibliographiques:} Lucas Terriel, \textit{Représenter et évaluer les données issues de la structuration et de la transcription automatique d'un corpus. L'exemple de la reconnaissance automatique des écritures manuscrites sur les répertoires de notaires du projet Lectaurep.}, mémoire de master \og Technologies numériques appliquées à l'histoire \fg{}, dir. Alix Chagué et Thibault Clérice, École nationale des chartes, 2020.

\clearpage
\thispagestyle{empty}
\cleardoublepage
\chapter*{Remerciements}
\addcontentsline{toc}{chapter}{Remerciements}
\markboth{Remerciements}{} 

Pour ce stage, qui s'est déroulé durant la période de confinement, je tiens à remercier l'ensemble des personnes qui m'ont aidé et soutenu dans cette situation particulière et qui m'ont encouragé dans mon travail.

En premier lieu, je remercie mon tuteur pédagogique, M. Thibault Clérice, et ma tutrice professionnel, Mme Alix Chagué, ingénieure recherche et développement, pour leurs appuis et leurs conseils avisés. 

Je remercie l'ensemble de l'équipe du projet Lectaurep, et le personnel du Minutier central des notaires de Paris aux Archives nationales, Mme Marie-Françoise Limon-Bonnet, responsable du département, Mme Aurélia Rostaing, responsable du pôle instruments de recherche, M. Gaetano Piraino, responsable à la DMOASI, M. Danis Habib, chargé d'études documentaires, Mme Virginie Grégoire, secrétaire de documentation, M. Benjamin Davy, agent technique d'accueil, surveillance et magasinage, et Mme Anna Chéru, stagiaire phase 3. 

Je remercie l'équipe ALMAnaCH d'Inria, qui ont su créer les conditions favorables d'accueil de mon stage et permis un environnement de travail stimulant; je remercie en particulier M. Laurent Romary, directeur de recherche, pour nos échanges et ses conseils pertinents et Mme Florianne Chiffoleau, ingénieure recherche et développement, pour son aide.

Comme le stage est un travail d'équipe avant tout, je remercie Jean-Damien Généro, collègue de master TNAH à l'École nationale des chartes, qui était stagiaire durant la même période à ALMAnaCH sur le projet Time us, dont la mise en commun de nos efforts et nos échanges sur les scripts ont permis le bon déroulement technique et pratique de nos stages respectifs.

Enfin je remercie ma famille et mes amis, pour leurs indéfectible soutient tout au long de cette période. 
\chapter*{Liste des sigles et abréviations}
\addcontentsline{toc}{chapter}{Liste des sigles et abréviations}
\markboth{Liste des sigles abréviations}{} 

\begin{itemize}
    \item A.N. : Archives Nationales
    \item DMC : Département du Minutier central des notaires de Paris  
    \item DMOASI : département de la maîtrise d’ouvrage du système d’information (direction de l’appui scientifique) 
\end{itemize}

\begin{center}
$\star$
\end{center} 

\begin{itemize}
    \item ALMAnaCH : \emph{Automatic Language Modelling and Analysis \& Computational Humanities}
    \item ANR : Agence Nationale de la Recherche
    \item EPI : Équipe-Projet Inria
    \item INRIA : Institut Nationale de Recherche en Informatique et Automatique
    \item LECTAUREP : Lecture automatique des répertoires
    \item W3C : World Wide Web Consortium
\end{itemize}

\begin{center}
$\star$
\end{center} 

\begin{itemize}
    \item EN : Entités Nommées
    \item ML : \emph{Machine Learning}
    \item REN : Reconnaissance d'Entités Nommées
    \item RNN : \emph{Recurrent neural network} - Réseau de neurones récurrents
    \item TAL : Traitement Automatique des Langues
\end{itemize}


\begin{center}
$\star$
\end{center} 

\begin{itemize}
    
    \item CSS : \emph{Cascading Style Sheets}
    \item CSV : \emph{Comma-separated values}
    \item DTD : \emph{Document Type Definition}
    \item HTML : \emph{HyperText Markup Language}
    \item HTR : \emph{Handwritten Text Recognition}
    \item OCR : \emph{Optical Character Recognition}
    \item ODD : \emph{One Document Does it all}
    \item PDF : \emph{Portable Document Format}
    \item RELAXNG : \emph{Regular Language for XML Next Generation}
    \item TEI : \emph{Text Encoding Initiative}
    \item XML : \emph{eXtensible Markup Language}
    \item XSLT : \emph{eXtensible Stylesheet Language Transformations}
\end{itemize}
\chapter*{Bibliographie}
\addcontentsline{toc}{part}{Bibliographie}
\markboth{Bibliographie}{Bibliographie}
\nocite{*}

\printbibliography[keyword={Humanités numériques},title={Humanités numériques}]

% Pour les parties principales : 
\mainmatter
\part*{Introduction}
\addcontentsline{toc}{part}{Introduction}
\markboth{Introduction}{Introduction}
\part{Le projet Lectaurep : un cas d'application de l' \og intelligence artificielle\fg{}  aux documents historiques}
\chapter{État de l'art du \textit{machine learning}, de la reconnaissance automatique des écritures manuscrites et de l'implication de ces techniques dans le cadre de projets de traitement automatique du langage (TAL)}
\section{Principes élémentaires du \textit{machine learning}}
\subsection{a voir ?}
\section{La reconnaissance automatique des écritures manuscrites : un domaine entre le traitement automatique du langage (TAL) et le \textit{machine learning}}
\chapter{Lectaurep, un projet de recherche et développement en reconnaissance automatique des écritures manuscrites}
\section{Des origines à la phase 3}
\section{Une dimension expérimentale}
\part{Représenter, enrichir et homogénéiser dans un format pivot les données et métadonnées au sein de la chaîne de traitement Lectaurep}

Objectifs de la mission ? 

\chapter{Enjeux et problématiques liés aux données et aux formats dans le projet}
\section{De l'importance des données et métadonnées}
\section{Les répertoires de notaires ne sont pas que des images numérisées !}
- permettre IIIF

\chapter{Le choix du XML TEI (\textit{Text Encoding Initiative}) comme format pivot}
\section{Un choix réaliste ? le format XML TEI dans d'autres projets et apports pour Lectaurep}
\section{Esquisse d'un format pivot et premières spécifications TEI grâce l'ODD}

\chapter{Simuler la récupération, l'export et la validation d'un format pivot XML TEI}
\section{Objectifs et buts de la simulation}
- ne parle forcément aux archivistes;
- besoin de visualiser les données dans un canevas TEI et d'envisager;
\section{Un CLI en Python pour générer un format pivot XML TEI et valider grâce à un schéma RELAX NG}
\part{Évaluer et contrôler la transcription sur des sets d'images comparés : proposer un vue synthétique des performances d'un modèle de transcription}

Objectifs de la mission ? 

\chapter{État de l'art pour l'évaluation des modèles de transcription entraînés avec le système OCR Kraken}
\section{Banc d'essai des outils existants : limites et avantages}
\section{Les métriques pour évaluer la transcription en question : définitions et recherche}

\chapter{Le développement d'une application : Kraken-Benchmark}
\section{Modélisation}
\section{Conception}
\section{Perspectives d'amélioration pour l'application}

\chapter{Tests de Kraken-Benchmark sur les images de répertoires de notaires}
\section{Préparation du corpus et mise en place des tests}
\section{Déroulement et résultats des tests}
\section{Un bilan mitigé ? des propositions pour améliorer les scores}
\part*{Conclusion}
\addcontentsline{toc}{part}{Conclusion}
\markboth{Conclusion}{Conclusion}

-> Meilleure préparation des données 
-> données plus hétérogènes
-> fomat pivot XML TEI avec un schéma plus complet

- Conditions vers le TAL


- Stage Lectaurep : dialogue entre le chercheur, le métier (archiviste) et les technologies numériques 
- Dans la peau d'un ingénieur en humnités numériques apport des réflexions techniques, des avantages de tel format, modélisation d'environnement et conception 

Partie II : 
- reste intégration dans eScriptorium 
- une fois le serveur IIIF configuré tester l'agrégation des liens IIIF dans le format pivot
- réfléchir à un référentiel des mains

Partie III : 

- mattermost et session de code d'Alix sur Python chaque semaine 2 rdv réguliers

Perspectives professionnelles : 
- Recrutement en Novembre => continuation sur Lectaurep 


Pour l'entrainement d'un segmenteur, il m'a par exemple dit qu'il fallait compter 300 à 400 images segmentées et au moins 100 époques d'entrainement pour avoir quelque chose de vraiment efficace
11:09 AM
donc pour LECTAUREP, ça veut dire qu'on a encore du boulot !

% Pour les annexes :
% exemples d'IR EAD et EAC
% exemples d'image de répertoire


\appendix
\part*{Annexes}
\addcontentsline{toc}{part}{Annexes}
\pagestyle{myheadings}
\markboth{Annexes}{Annexes}

Les annexes présentent à la fois les livrables effectués durant le stage et des compléments. Cette partie contient également les chemins de localisation des fichiers. Ils sont reproduits sur une \citecode{clé usb} et sur un dépôt Github accessible à l'adresse suivante : \url{https://github.com/Lucaterre/L-TERRIEL_memoireDeStage_M2TNAH_ENC}.

\chapter{Sources et Ecosystème Lectaurep}

localisation : \citecode{/A-Sources\_et\_Ecosystème\_Lectaurep/} contenant :

\section{histoire du projet lectaurep}
localisation : \citecode{/A1-histoire\_projet\_lectaurep/} contenant :
\begin{itemize}
    \item \citecode{note}
\end{itemize}

% pdf des différentes présentation

\section{Extraits du corpus des répertoires de notaires}
localisation : \citecode{A2-Extraits\_du\_corpus\_des\_répertoires\_de\_notaires/} contenant :
\begin{itemize}
    \item \citecode{image\_repertoire\_de\_notaire}
    \item \citecode{exemple de la structuration en colonnes} \\
        \citecode{des répertoires proposé par Marie-Laurence Bohnomme}
\end{itemize}

% inclure les figures

\section{Outils généraux utilisés dans Lectaurep}

localisation : \citecode{A3-Outils\_de\_Lectaurep} contenant :
\begin{itemize}
    \item \citecode{Interface\_eScriptorium.png}
    \item \citecode{Golden\_random\_set\_sharedocs.png}
    \item \citecode{Entrainement\_modele\_cluster\_calcul\_RIOC.png}
\end{itemize}
% inclure les figures
\chapter{Format pivot XML TEI Lectaurep}
localisation : \citecode{/B-Format\_pivot\_XML\_TEI\_Lectaurep} contenant :
\skip
\dirtree{%
.1 /B-Format\_pivot\_XML\_TEI\_Lectaurep/.
   .2 Doc/\DTcomment{regroupe la documentation sur le projet de format pivot XML TEI pour Lectaurep}.
        .3 Modélisation\_et\_documentation\_format\_pivot/.
           .4 template\_pivot\_TEI\_lectaurep.xml\DTcomment{Canevas pour formaliser les attentes et les réflexions des acteurs pour l'inclusion des données Lectaurep dans la TEI.}.
           .4 Les ODD aux formats : XML, PDF et HTML.\DTcomment{La documentation standard TEI pour le format pivot TEI Lectaurep.}.
           .4 oddbyexample.xsl\DTcomment{Une feuille de transformation XSL pour générer une nouvelle ODD à partir de la template pivot.}.
        .3 Crosswalks\_vers\_TEI\DTcomment{Un ensemble de transformations XSL utiles vers les spécifications TEI et ALTO, issues de différents projets.}.
            .4 ALTO -> TEI.
            .4 EAD -> TEI (2 versions).
            .4 ALTO -> TEI.
            .4 PAGE -> ALTO.
   .2 Generator\_Lectaurep2TEI/\DTcomment{CLI Python permettant de simuler 
	                                     la conversion des données issues de Lectaurep (EAD-EAC, ALTO, EXIF) 
	                                     vers un format XML TEI Pivot suivant les recommandations de l'ODD 
	                                     (voir le readme.md pour plus de détails).}.
        .3 Output/\DTcomment{Dossier dans lequel est généré la sortie du script.}.
            .4  test\_legay\_tei.xml\DTcomment{Format pivot TEI Lectaurep de test généré en sortie du script.}.
        .3 Sets\_test\_Legay/\DTcomment{Contient un ensemble de fichiers correspondant aux données fournies par Lectaurep pour tester le script Lectaurep2TEI et généré une première version du format pivot.}.
            .4  Data\_xml\_alto/\DTcomment{fichiers XML ALTO correspondants aux transcriptions vérité terrain récupérés sur ShareDocs relatives aux images numérisées du répertoire de notaire d'Ernest Legay (étude XXIII).}.
            .4  Data\_xml\_ead\_eac/\DTcomment{fichiers XML EAD et XML EAC-CPF correpondants aux instruments de recherche des répertoires du notaire Ernest Legay (étude XXIII) et notices producteurs récupérés sur la Salle des inventaires virtuels des Archives nationales.}.
                .5 FRAN\_IR\_041698.xml \DTcomment{IR \inquote{Minutes et répertoires du notaire Ernest LEGAY, 25 février 1875 - 14 mai 1902 (étude XXIII)}.}.
                .5 FRAN\_IR\_051379.xml \DTcomment{IR \inquote{Images des répertoires du notaire Ernest Legay pour l'étude XXIII}.}.
                .5 FRAN\_NP\_010150.xml \DTcomment{Notice producteur de l'étude XXIII.}.
                .5 FRAN\_NP\_010150.xml \DTcomment{Notice producteur de Legay, Ernest.}.
                .5 Schémas XML et DTD EAD et EAC-CPF.
            .4  images\DTcomment{Images numérisées du répertoire de notaire d'Ernest Legay (étude XXIII) issues du Golden Set de ShareDocs.}.
        .3 generator\_utils/.\DTcomment{Ensemble des modules Python utiles au fonctionnement du CLI generator Lectaurep2TEI. Pour les usages consulter les \textit{docstrings}.}.
            .4  \_\_init\_\_.py.
            .4  build\_utils.py.
            .4  extract\_utils.py.
            .4  validation\_utils.py.
        .3 pack\_schemaRNG/.\DTcomment{Doit accueillir à terme les schémas de validation Relax NG du format pivot XML TEI Lectaurep.}.
            .4 tei\_all.rng.\DTcomment{Schéma Relax NG TEI ALL.}.
        .3 Lectaurep\_ALTO2TEI.xsl.\DTcomment{Une feuille de transformation XSL ALTO vers TEI nécéssaire pour le fonctionnement du script principal. Pour les usages consulter la \textit{docstring}.}.
        .3 Lectaurep\_EADEAC2TEI.xsl.\DTcomment{Une feuille de transformation XSL EAD/EAC vers TEI nécéssaire pour le fonctionnement du script principal. Pour les usages consulter la \textit{docstring}.}.
        .3 catalog\_alto.xml.\DTcomment{Fichier automatiquement créé par le script, nécéssaire pou l'usage de la fonction Xpath 2.0 \citecode{collection()} pour la feuille XSL \citecode{Lectaurep\_ALTO2TEI.xsl}}.
        .3 catalog\_ead\_eac.xml.Lectaurep\_ALTO2TEI.xsl\DTcomment{Fichier automatiquement créé par le script, nécéssaire pou l'usage de la fonction Xpath 2.0 \citecode{collection()} pour la feuille XSL \citecode{Lectaurep\_EADEAC2TEI.xsl}}.
        .3 generator\_Lectaurep2TEI\_logo.png.
        .3 inr\_logo\_grisbleu.png.
        .3 main.py.\DTcomment{Script Python principal d'exécution du CLI generator Leactaurep2TEI.}.
        .3 readme.md.\DTcomment{Documentation pour installer et lancer le programme.}.
        .3 requirements.txt.\DTcomment{Ensemble des \textit{packages} Python nécessaires à l'utilisation du CLI}.
        .3 snap\_generator.png.
   .
}

\chapter{Application Kraken Benchmark}
localisation : \citecode{/C-Application\_Kraken\_Benchmark} contenant :
\skip
\dirtree{%
.1 /C-Application\_Kraken\_Benchmark/.
   .2 Documentation-Reasearch/.\DTcomment{Contient les versions du \textit{notebook Jupyter} exposant les réflexions sur les métriques et les algorithmes utilisés dans l'application Kraken-Benchmark.}.
    .3 Evaluation de la similarité entre deux séquences dans le contexte de la reconnaisance automatique de caractères.\DTcomment{\textit{notebook Jupyter} en versions PDF, HTML et IPYNB (format natif).}. 
    .3 Ensemble d'images rattachés au \textit{notebook Jupyter}.
   .2 KB-app/.\DTcomment{Dossier contenant les fichiers pour faire fonctionner l'application Kraken-Benchmark.}.
    .3 STS\_Tools/.\DTcomment{Le \textit{package} Python \textit{Sequences to Similarity} créée pour l'application Kraken-Benchmark contient deux modules Python.}.
        .4 STSig.py.\DTcomment{module Python \textit{Sequences To Signals} (en cours de développement); expérience pour visualiser et comparer deux chaînes de caractères sous la forme de signaux.}.
        .4 SynSemTS.py.\DTcomment{module Python \textit{Syntactic Semantic To Similarity} qui contient la plupart des métriques (syntaxiques et sémantiques) utilisé dans l'application pour l'analyse deux chaînes de caractères correspondant à la vérité terrain et la transcription issue du système HTR.}.
        .4  \_\_init\_\_.py.
    .3 kb\_report/.\DTcomment{Dossier contenant les fichiers pour la gestion de la partie affichage dans le navigateur de l'application en \textit{Flask}.}.
        .4  static/.\DTcomment{contient les images de l'application à afficher dans le navigateur.}.
        .4  templates/.\DTcomment{contient les pages HTML de l'application.}.
        .4  \_\_init\_\_.py.
        .4  routing.py\DTcomment{Script Python qui contient les différentes routes URL de l'application.}.
    .3 kb\_utils/.\DTcomment{Dossier contenant un module Python nécessaire au fonctionnement de l'application.}.
        .4  \_\_init\_\_.py.
        .4  kb\_utils.py.\DTcomment{Module Python contenant des fonctions utiles au fonctionnement de l'application.}.
    .3 environment.yml.\DTcomment{Fichier pour la création d'un environnement virtuel \textit{Conda} contenant les \textit{packages} Python nécéssaires.}.
    .3 kraken\_benchmark.py.\DTcomment{Script principal pour l'exécution de l'application Kraken-Benchmark.}.
   .2 sets\_test/.\DTcomment{Jeux de fichiers pour effectuer des tests dans Kraken-Benchmark et scripts Python.}.
    .3 jules\_verne\_set\_test/.\DTcomment{Jeux de données utilisés pour tester l'application au fur et à mesure des développements.}.
        .4  images/.\DTcomment{Contient des numérisations (formats \citecode{.jpeg}) de l'ouvrage \textit{Voyage au centre de la terre} récupérés sur Gallica.}.
        .4  dataset\_GT/.\DTcomment{Contient les transcriptions vérités terrains en format texte brut utilisées pour comparer les résultats issus de l'HTR de Kraken-Benchmark. Édités à partir du CLI Kraken.}.
        .4  model/.\DTcomment{Contient le modèle (format \citecode{.mlmodel}) entraîné sur le CLI Kraken pour réaliser l'OCR dans Kraken-Benchmark sur le set de numérisations de \textit{Voyage au centre de la terre}.}.
    .3 sets\_tests\_lectaurep/.\DTcomment{Jeux de données utilisés pour tester les modèles de transcription issus du CLI Kraken sur des images de répertoires de notaires sélectionnés pour leurs particularismes (pour plus de détails sur les sets d'images et les modèles HTR utilisés voir le fichier \citecode{CR\_tests\_lectaurep\_KB.md}) pour évaluer la qualité des transcriptions durant le stage.}.
        .4  different\_control\_set/.\DTcomment{Contient les transcriptions vérité terrains (\citecode{GT} et les images).}.
        .4  homogeneous\_control\_set/.\DTcomment{Contient les transcriptions vérité terrains (\citecode{GT} et les images).}.
        .4  set\_material\_defects/.\DTcomment{Contient les transcriptions vérité terrains (\citecode{GT} et les images).}.
        .4  set\_writing\_defects/.\DTcomment{Contient les transcriptions vérité terrains (\citecode{GT} et les images).}.
        .4  snaps\_tests\_lectaurep/.\DTcomment{Contient des captures sous la forme de fichier HTML de l'application Kraken-Benchmark réalisés lors des différents tests.}.
            .5 report\_html/.\DTcomment{Contient des captures de la page d'accueil de Kraken-Benchmark avec les principales métriques.}.
            .5 versus\_text\_html/.\DTcomment{Contient des captures de la fonctionnalité de comparaison de la vérité terrain et de la transcription HTR dans Kraken-Benchmark.}.
        .4  models/.\DTcomment{Contient les modèles HTR, entraînés avec le CLI Kraken, et utilisés sur les différents jeux de données.}.
        .4  CR\_tests\_lectaurep\_KB.md\DTcomment{Compte-rendu présentant le déroulement des tests, la description des sets de tests, des modèles et des résultats des expériences.}.
        .4  details\_data\_average\_tests\_model\_test\_lectaurep\_bin\_accuracy\_6064.mlmodel - Feuille 1.pdf\DTcomment{Fichier PDF présentant la moyenne des résultats des tests avec le modèle 6064, utilisé pour le graphique radar.}.
        .4   details\_data\_average\_tests\_model\_test\_lectaurep\_bin\_accuracy\_8164.mlmodel - Feuille 1.pdf\DTcomment{Fichier PDF présentant la moyenne des résultats des tests avec le modèle 8164, utilisé pour le graphique radar.}.
        .4   radar\_test\_II\_model\_0-8164.png.
        .4  radar\_test\_I\_model\_0-6064.png.
    .3 alto2text.py.\DTcomment{Script Python adapté, provenant de la \textit{StaatsbibliothekBerlin} permettant la conversion de certaines transcriptions vérités terrains en XML ALTO vers des fichiers texte brut.}.
    .3 radar\_graph.py.\DTcomment{Script Python adapté permettant la génération d'un graphique radar, utile pour le rapport concernant les tests spécifiques à Lectaurep durant le stage.}.
   .2 README.md.\DTcomment{Présentation et documentation pricipale de l'application pour l'installation et l'utilisation.}.
   .
}

\backmatter
\chapter*{Lexique informatique}
\addcontentsline{toc}{chapter}{Lexique informatique}
\markboth{Lexique informatique}{} 

\textit{\small{liste non exhaustive}}

\begin{itemize}
    \item \textbf{Algorithme} : Suite d'étapes réalisées par un programme informatique pour effectuer une tâche.
    \item \textbf{\textit{Back-office/Front-office}} : Dans une application désigne la partie visible par le client (\textit{front-office}) et la partie qui concerne les systèmes d'informations (bases de données) et leurs gestion invisible de l'utilisateur final (\textit{back-office}).
    \item \textbf{\textit{Blog}} : Type de site \textit{web} qui permet la publication périodique d'articles scientifiques rendant compte de l'actualité d'un projet ou d'une thématique.
    \item \textbf{Interface en ligne de commande} : CLI ou \textit{Command Line Interface} en anglais - c'est une interface homme machine dépourvue d'aspect graphique et où la communication s'effectue en mode texte grâce à des lignes de commande (texte sur le clavier) pour demander à l'ordinateur d'effectuer une opération.
    \item \textbf{\textit{Commit}} : Dans un système de versionnage, commande qui permet de valider des modifications locales vers un référentiel central afin de les mettre à disposition.
    \item \textbf{Depôt (informatique)} : \textit{repository} en anglais -  S'applique aux logiciels de gestion de versions, stockage organisé de données; endroit où l'on dépose le code-source.
    \item \textbf{\textit{Docstring}} : Chaîne de caractères pour documenter un segment spécifique du code informatique.
    \item \textbf{Fonction} : En programmation, \inquote{sous-programme} qui permet de réaliser des tâches répétitives pour alléger du code.
    \item \textbf{Interface graphique} : GUI ou \textit{Graphical User Interface} en anglais - est un environnement qui permet l'interaction de l'homme et la machine comme la métaphore du bureau dans la plupart des systèmes d'exploitation.
    \item \textbf{Interpréteur de commande} : Terminal informatique, c'est un logiciel étant compris de base dans le système d'exploitation, il permet d'interpréter des commandes d'un utilisateur ou d'une utilisatrice dans un environnement dépourvu d'interface graphique.
    \item \textbf{\textit{Issue}} : Dans une plateforme de versionnage, sorte de billet qui permet d'émettre des suggestions ou qui fait état des bugs dans un contexte de développement. 
    \item \textbf{\textit{Merge}} : Action de fusionner des branches (versions) différentes d'un dépôt informatique. (Cf. \textit{pull request} (Github) ou \textit{merge request} (Gitlab))
    \item \textbf{module} : En développement, fichier contenant des fonctions, des classes ou des variables pouvant être importé dans un \textit{script} pour en utiliser le contenu.
    \item \textbf{\textit{Package}} - sym. \textit{library} : Ensemble de modules de traitements spécifiques pouvant être importé.
    \item \textbf{\textit{Parser}} :  Programme informatique qui permet l'analyse syntaxique des éléments pour leurs donner une signification. 
    \item \textbf{\textit{Push}} : Dans un système de versionnage, commande qui après un \textit{commit} permet d'envoyer les modifications d'un système local vers un référentiel central pour partager ces dernières.
    \item \textbf{\textit{Open-source}} : Communauté et types de licences qui s'appliquent à des programmes ou à des logiciels qui permettent la redistribution et la réutilisation de leurs code-sources.  
    \item \textbf{\textit{Script}} : Programme ou extrait de programme qui permet de réaliser une tâche prédéfinie (Cf. Algorithme). 
    \item \textbf{Système d'exploitation} : \textit{Operating System} (OS) en anglais - Ensemble de programmes qui permet d'utiliser les ressources d'un ordinateur. Le logiciel système pilote les ressources matérielles de l'ordinateur et reçoit les instructions des usagers ou d'autres logiciels.
    \item \textbf{Logiciel de gestion de versions} : Logiciel qui permet de stocker des fichiers en conservant la chronologie de l'ensemble des modifications (versions) qui y ont été effectuées.
    \item \textbf{Tests unitaires} : En programation informatique, procédure qui permet de vérifier une partie précise d'un logiciel ou d'un programme pour s'assurer de son fonctionnement.
\end{itemize}
\listoffigures
\tableofcontents

\cfoot{Ce document a été rédigé en \LaTeX via l'interface en ligne Overleaf}

\end{document}