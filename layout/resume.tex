\chapter*{Résumé}
\addcontentsline{toc}{chapter}{Résumé}
\markboth{Résumé}{} 
Ce mémoire a été réalisé en vue de l'obtention du diplôme de Master 2 \inquote{Technologies numériques appliquées à l'histoire} de l'École nationale des chartes. Il a été rédigé à la suite d'un stage de quatre mois au sein de l'équipe ALMAnaCH d'Inria, et dont le déroulé s'est inscrit dans le cadre de Time Us. Ce projet de recherche pluri-institutionnel porte sur l'histoire de l'industrie du textile en France (fin XVII\up{e}-début XX\up{e} siècles) et sur la reconstitution des budget temps des ouvriers et ouvrières du textile. Time Us explore également l'utilisation d'outils informatique pour réaliser cette recherche. 
Ce mémoire étudie le traitement informatique appliqué à une partie du corpus de Time Us, de la structuration à l'éditorialisation. 
Il s'agit d'une analyse critique des enjeux, stratégies et résultats envisagés dans le cadre du projet Time Us autant que du stage, dont le but est de rendre compte d'un exemple de projet et de développement s'inscrivant dans le cadre des humanités numériques.


\bigskip

%informations à compléter
\textbf{Mots-clefs:} TAL; métadonnées; données; format pivot; XML-TEI; développement applicatif; similarité syntaxique; similarité sémantique; métriques \textit{text-to-text}; OCR; HTR; Machine learning; Intelligence artificielle; répertoires de notaires; valorisation patrimoniale; Humanités numériques.

\bigskip
\bigskip
\bigskip

% informations à compléter
\textbf{Informations bibliographiques:} Lucas Terriel, \textit{Représenter et évaluer les données issues de la structuration et de la transcription automatique d'un corpus. L'exemple de la reconnaissance automatique des écritures manuscrites sur les répertoires de notaires du projet Lectaurep.}, mémoire de master \og Technologies numériques appliquées à l'histoire \fg{}, dir. Alix Chagué et Thibault Clérice, École nationale des chartes, 2020.

\clearpage
\thispagestyle{empty}
\cleardoublepage